%Presentation for Python Documentationa
\documentclass{beamer}
\usetheme{metropolis}           % Use metropolis theme

\usepackage{url}
\usepackage{hyperref}
\definecolor{links}{HTML}{7c2e43}
\hypersetup{colorlinks,linkcolor=,urlcolor=links}
\usepackage{minted}
\usepackage{csquotes}

\title{Documentation for Research Coding Projects}
\date{\today}
\author{Matthew McKay}
\institute{QuantEcon - Research School of Economics, ANU, Australia}

\begin{document}
\maketitle

\section{Python}

\begin{frame}{Table of Contents}
\begin{itemize}
    \item 'Good' Code Style (PEP 8)
    \item Docstrings (PEP 257)
    \item Jupyter and Introspection
    \item Documentation Standards
        \begin{enumerate}
            \item PEP 257
            \item numpydoc
            \item googledoc
        \end{enumerate}
    \item Sphinx Compilation System
    \item QuantEcon Project Documentation (Example)
    \item Resources and Links
\end{itemize}    
\end{frame}

\begin{frame}{'Good' Code Style (PEP 8)}

Emphasis on Readability

This is good for \textbf{others} and good for \textbf{future self}

Python \href{https://www.python.org/dev/peps/pep-0008/}{PEP 8}

\begin{enumerate}
\item provides a set of rules and a standard for writing code
\item widely adopted by the Python Community
\item emphasis is on readability of code
\end{enumerate}

Many editors assist with easily conforming to these types of rules such as auto-indentation

\end{frame}

%START SECTION

\begin{frame}{\href{https://www.python.org/dev/peps/pep-0008/\#code-lay-out}{Code Layout}}

Summary

\begin{enumerate}
    \item \href{https://www.python.org/dev/peps/pep-0008/\#indentation}{Indentation}
    \item \href{https://www.python.org/dev/peps/pep-0008/\#tabs-or-spaces}{Tabs or Spaces}
    \item \href{https://www.python.org/dev/peps/pep-0008/\#maximum-line-length}{Maximum Line Length}
    \item \href{https://www.python.org/dev/peps/pep-0008/\#should-a-line-break-before-or-after-a-binary-operator}{Should a line break before or after a binary operator?}
    \item \href{https://www.python.org/dev/peps/pep-0008/\#blank-lines}{Blank Lines}
    \item \href{https://www.python.org/dev/peps/pep-0008/\#source-file-encoding}{Source File Encoding}
    \item \href{https://www.python.org/dev/peps/pep-0008/\#imports}{Importing}
\end{enumerate}

\end{frame}

\begin{frame}{\href{https://www.python.org/dev/peps/pep-0008/\#naming-conventions}{Naming Conventions}}

Summary

\begin{enumerate}
    \item \href{https://www.python.org/dev/peps/pep-0008/\#class-names}{Class Names} are CapCase
    \item \href{https://www.python.org/dev/peps/pep-0008/\#function-names}{Function Names} are lowercase with words separated by an underscore
    \item \href{https://www.python.org/dev/peps/pep-0008/\#function-and-method-arguments}{Function and Method Arguments}
    \item \href{https://www.python.org/dev/peps/pep-0008/\#constants}{Constants} are all CAPS
\end{enumerate}

\end{frame}

\begin{frame}{\href{https://www.python.org/dev/peps/pep-0008/\#comments}{Comments}}

Summary

\begin{enumerate}
\item \href{https://www.python.org/dev/peps/pep-0008/\#block-comments}{Block Comments}
\item \href{https://www.python.org/dev/peps/pep-0008/\#inline-comments}{Inline Comments}
\item \href{https://www.python.org/dev/peps/pep-0008/\#documentation-strings}{Documentation Strings}
\end{enumerate}

\end{frame}

%END

%START SECTION

\begin{frame}{Docstrings and PEP 257}

A docstring is a string literal that occurs as the first statement in a module, function, class, 
or method definition. Such a docstring becomes the \mintinline{python}{__doc__} special attribute of that object.

  For consistency:

  \begin{enumerate}
    \item always use """triple double quotes""" around docstrings \textbf{(Most Common)}
    \item Use r"""raw triple double quotes""" if you use any backslashes in your docstrings 
    \item Use u"""Unicode triple-quoted strings""" for unicode docstrings
  \end{enumerate}

\end{frame}

\begin{frame}{PEP 257}

The initial standard for writing docstrings in Projects

\begin{enumerate}
\item \href{https://www.python.org/dev/peps/pep-0257/\#one-line-docstrings}{One Line Docstrings}
\item \href{https://www.python.org/dev/peps/pep-0257/\#multi-line-docstrings}{Multi Line Docstrings}
\item \href{https://www.python.org/dev/peps/pep-0257/\#handling-docstring-indentation}{Handling Docstring Indentation}
\end{enumerate}

\end{frame}

\begin{frame}[fragile]{Docstring Example: One-line}

  One-line type docstrings

  \begin{minted}[fontsize=\small]{python}
  def add(a, b):
      """add two numbers together"""
      return a + b
  \end{minted}

the docstring is attached to the add function as the doc attribute \mintinline{python}{add.__doc__}

\end{frame}

\begin{frame}[fragile]{Docstring Example: Multi-line}

  \begin{minted}[fontsize=\small]{python}
  def add(a, b, show=True):
    """
    add two numbers together

    Keyword arguments:
    show -- boolean, show the resulting value (default=True)
    """
    result = a + b 
    if show: 
        print("Result = {}".format(result))
    return result
  \end{minted}

\end{frame}


\begin{frame}{Class Docstrings}

\end{frame}

%END

\begin{frame}{\href{http://jupyter.org/}{Jupyter} and Introspection}

Jupyter

\begin{enumerate}
  \item as an environment, and 
  \item Jupyter as a tool for introspection
\end{enumerate}

\end{frame}

\begin{frame}{Jupyter as a documentation tool ...}

Jupyter itself can be useful for documenting research projects as it provides modal cells:

\begin{enumerate}
  \item prose (with LaTeX math markup support)
  \item programming
  \item data, visualization and plotting
\end{enumerate}

\end{frame}

\begin{frame}[fragile]{Useful feature of Jupyter ...}

Jupyter (via IPython) has a number of useful query features for objects

\begin{minted}[fontsize=\small]{python}
import pandas as pd 

df = pd.DataFrame([1,2,3,4,5])

df.<tab>            #Provides access to object methods
df.sort(<tab>)      #Provides method signature
df.sort(<tab><tab>) #Provides full docstring
df.sort?            #Provides docstring in new window
\end{minted}

\end{frame}

%Documentation Standards

\begin{frame}{Why use Documentation Standards?}

\textbf{Main:} A lot of thought has gone into design and readability

\textbf{Others:}

\begin{enumerate}
  \item Easier to work across projects in a community
  \item Integrates with software that can build useful userguides, notes, or manuals
  \item Layouts can be more geared towards a specific community type (Scientific Community = SciPy/NumPy)
\end{enumerate}

\end{frame}

\begin{frame}{\href{https://github.com/numpy/numpy/blob/master/doc/HOWTO_DOCUMENT.rst.txt#docstring-standard}{Numpy/Scipy Style}}  

\textbf{Focus:} layout that can produce a well-formatted reference guide

Uses a subset of re-structured text (RST) markup:
\begin{enumerate}
  \item maintain readability in text editors
  \item allows for more advanced formatting to be inferred
  \item allows the use of LaTeX for math
  \item bibtex citations
  \item use of sphinx directives to add warnings, notes etc.
\end{enumerate}

A reStructured text primer can be found \href{http://www.sphinx-doc.org/en/stable/rest.html}{here}

\end{frame}

\begin{frame}{Numpy Docstrings: Basic Structure}



\end{frame}

\begin{frame}{Google Style}

\end{frame}

%END Documentation Standards

\begin{frame}{\href{http://www.sphinx-doc.org/en/stable/}{Sphinx} and Compilation Systems}

A benefit of using a standard for documentation is nice integration with build / compilation systems such as \href{http://www.sphinx-doc.org/en/stable/}{Sphinx}

You can easily document projects using \href{http://www.sphinx-doc.org/en/stable/ext/autodoc.html}{autodoc}

\end{frame}

\begin{frame}{QuantEcon Project Documentation (Example)}

QuantEcon.py documentation can be found: \url{http://quanteconpy.readthedocs.io/en/latest/}

\end{frame}

\begin{frame}{Resources}
\begin{enumerate}
\item \href{https://www.python.org/dev/peps/pep-0008/}{Python - PEP8}
\item \href{https://www.python.org/dev/peps/pep-0257/}{Python - Docstring Conventions}
\item \href{https://github.com/numpy/numpy/blob/master/doc/HOWTO_DOCUMENT.rst.txt}{NumpyDoc how-to Guide}
\item \href{http://sphinxcontrib-napoleon.readthedocs.io/en/latest/example_google.html}{Google Style Python Docstrings}
\item \href{http://www.sphinx-doc.org/en/stable/}{Sphinx}
\begin{itemize}
  \item \href{http://sphinxcontrib-napoleon.readthedocs.io/en/latest/sphinxcontrib.napoleon.html}{sphinxcontrib.napoleon package}
  \item \href{https://github.com/numpy/numpydoc}{Numpydoc}
  \item \href{http://www.sphinx-doc.org/en/stable/ext/autodoc.html}{autodoc}
\end{itemize}
\item \href{http://jupyter.org/}{Jupyter}
\end{enumerate}
\end{frame}

\section{Julia}

\begin{frame}{Julia Documentation}

Current state of the art: \href{https://github.com/JuliaDocs/Documenter.jl}{Documenter.jl}

Features:
\begin{enumerate}
  \item supports markdown
  \item support for LaTeX math
  \item Doctests, cross-references for docs, linter etc.
\end{enumerate}

documentation standards still to solidify

\end{frame}

\begin{frame}[fragile]{Julia Example}

  \begin{minted}[fontsize=\small]{julia}
    """
    add two numbers together

    # Keyword arguments:
    * `show:boolean`: show the resulting value (default=true)   
    """
    function add(a, b; show=true)
       result = a + b
       if show == true
         println("Result = ", result)
       end
       return result
     end

  \end{minted}

\end{frame}

\end{document}