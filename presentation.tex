%Presentation for Python Documentationa
\documentclass{beamer}
\usetheme{metropolis}           % Use metropolis theme

\usepackage{url}
\usepackage{hyperref}
\definecolor{links}{HTML}{7c2e43}
\hypersetup{colorlinks,linkcolor=,urlcolor=links}
\usepackage{minted}
\usepackage{csquotes}

\title{Documentation for Research Coding Projects}
\date{\today}
\author{Matthew McKay}
\institute{QuantEcon - Research School of Economics, ANU, Australia}

\begin{document}
\maketitle

\begin{frame}{Where to get this?}

\url{https://github.com/mmcky/anu.macroreadinggroup.code-documentation.git}

\end{frame}

\section{Python}

\begin{frame}{Table of Contents}
\begin{itemize}
    \item 'Good' Code Style (PEP 8)
    \item Docstrings (PEP 257)
    \item Jupyter and Introspection
    \item Documentation Standards
        \begin{enumerate}
            \item PEP 257
            \item numpydoc
            \item googledoc
        \end{enumerate}
    \item Sphinx Compilation System
    \item Resources and Links
\end{itemize}    
\end{frame}

\begin{frame}{'Good' Code Style (PEP 8)}

Emphasis on \textbf{Readability}

This is good for \textbf{others} and good for \textbf{future self}

Python \href{https://www.python.org/dev/peps/pep-0008/}{PEP 8}

\begin{enumerate}
\item provides a set of rules and a standard for writing code
\item widely adopted by the Python Community
\item emphasis is on readability of code
\end{enumerate}

Many editors assist with easily conforming to these types of rules such as auto-indentation

\end{frame}

%START SECTION

\begin{frame}{\href{https://www.python.org/dev/peps/pep-0008/\#code-lay-out}{Code Layout}}

Summary

\begin{enumerate}
    \item \href{https://www.python.org/dev/peps/pep-0008/\#indentation}{Indentation}
    \item \href{https://www.python.org/dev/peps/pep-0008/\#tabs-or-spaces}{Tabs or Spaces}
    \item \href{https://www.python.org/dev/peps/pep-0008/\#maximum-line-length}{Maximum Line Length}
    \item \href{https://www.python.org/dev/peps/pep-0008/\#should-a-line-break-before-or-after-a-binary-operator}{Should a line break before or after a binary operator?}
    \item \href{https://www.python.org/dev/peps/pep-0008/\#blank-lines}{Blank Lines}
    \item \href{https://www.python.org/dev/peps/pep-0008/\#source-file-encoding}{Source File Encoding}
    \item \href{https://www.python.org/dev/peps/pep-0008/\#imports}{Importing}
\end{enumerate}

\end{frame}

\begin{frame}{\href{https://www.python.org/dev/peps/pep-0008/\#naming-conventions}{Naming Conventions}}

Summary

\begin{enumerate}
    \item \href{https://www.python.org/dev/peps/pep-0008/\#class-names}{Class Names} are CapCase
    \item \href{https://www.python.org/dev/peps/pep-0008/\#function-names}{Function Names} are lowercase with words separated by an underscore
    \item \href{https://www.python.org/dev/peps/pep-0008/\#function-and-method-arguments}{Function and Method Arguments}
    \item \href{https://www.python.org/dev/peps/pep-0008/\#constants}{Constants} are all CAPS
\end{enumerate}

\end{frame}

\begin{frame}{\href{https://www.python.org/dev/peps/pep-0008/\#comments}{Comments}}

Summary

\begin{enumerate}
\item \href{https://www.python.org/dev/peps/pep-0008/\#block-comments}{Block Comments}
\item \href{https://www.python.org/dev/peps/pep-0008/\#inline-comments}{Inline Comments}
\item \href{https://www.python.org/dev/peps/pep-0008/\#documentation-strings}{Documentation Strings}
\end{enumerate}

\end{frame}

%END

%START SECTION

\begin{frame}{Docstrings and PEP 257}

A docstring is a string literal that occurs as the first statement in a module, function, class, 
or method definition. Such a docstring becomes the \mintinline{python}{__doc__} special attribute of that object.

  For consistency:

  \begin{enumerate}
    \item always use """triple double quotes""" around docstrings \textbf{(Most Common)}
    \item Use r"""raw triple double quotes""" if you use any backslashes in your docstrings 
    \item Use u"""Unicode triple-quoted strings""" for unicode docstrings
  \end{enumerate}

\end{frame}

\begin{frame}{PEP 257}

The initial standard for writing docstrings in Projects

\begin{enumerate}
\item \href{https://www.python.org/dev/peps/pep-0257/\#one-line-docstrings}{One Line Docstrings}
\item \href{https://www.python.org/dev/peps/pep-0257/\#multi-line-docstrings}{Multi Line Docstrings}
\item \href{https://www.python.org/dev/peps/pep-0257/\#handling-docstring-indentation}{Handling Docstring Indentation}
\end{enumerate}

\end{frame}

\begin{frame}[fragile]{Docstring Example: One-line}

  One-line type docstrings

  \begin{minted}[fontsize=\small]{python}
  def add(a, b):
      """add two numbers together"""
      return a + b
  \end{minted}

the docstring is attached to the add function as the doc attribute \mintinline{python}{add.__doc__}

\end{frame}

\begin{frame}[fragile]{Docstring Example: Multi-line}

  \begin{minted}[fontsize=\small]{python}
  def add(a, b, show=True):
    """add two numbers together

    Keyword arguments:
    show -- boolean, show the resulting value (default=True)
    """
    result = a + b 
    if show: 
        print("Result = {}".format(result))
    return result
  \end{minted}

\end{frame}


\begin{frame}[fragile]{Class Docstrings}

  \begin{minted}[fontsize=\small]{python}
  class TestAdd(object):
    """This is a class docstring"""
    
    def __init__(self, a, b, show=True):
        """
        add two numbers together and store the result
    
        Keyword arguments:
        show -- boolean, show the resulting value (default=True)

        .. note:: this is the __init__ docstring
        """
        self.result = a + b
  \end{minted}

  for documenting \mintinline{python}{__init__} method you typically pick a convention to use class or method docstring

\end{frame}

%END

\begin{frame}{\href{http://jupyter.org/}{Jupyter} and Introspection}

Jupyter

\begin{enumerate}
  \item as an environment for documenting research, and 
  \item Jupyter as a tool for introspection
\end{enumerate}

\end{frame}

\begin{frame}{Jupyter as a documentation tool ...}

Jupyter itself can be useful for documenting research projects as it provides modal type cells:

\begin{enumerate}
  \item prose (with LaTeX math markup support)
  \item programming
  \item data, visualization and plotting
\end{enumerate}

\end{frame}

\begin{frame}[fragile]{Useful feature of Jupyter ...}

Jupyter (via IPython) has a number of useful query features for objects

\begin{minted}[fontsize=\small]{python}
import pandas as pd 

df = pd.DataFrame([1,2,3,4,5])

df.<tab>            #Provides access to object methods
df.sort(<tab>)      #Provides method signature
df.sort(<tab><tab>) #Provides full docstring
df.sort?            #Provides docstring in new window
df.sort??           #Provides docstring and full set of code for method
\end{minted}

\end{frame}

%Documentation Standards

\begin{frame}{Why use Documentation Standards?}

\textbf{Main:} A lot of thought has gone into design and readability

\textbf{Others:}

\begin{enumerate}
  \item Easier to work across projects in a community
  \item Integrates with software that can build useful user-guides, notes, or manuals
  \item use markup (via RST) while retaining readability
  \item can be tailored to the needs of a specific community
\end{enumerate}

\end{frame}

\begin{frame}{\href{https://github.com/numpy/numpy/blob/master/doc/HOWTO_DOCUMENT.rst.txt\#docstring-standard}{Numpy/Scipy Style Docstrings}}  

\textbf{Focus:} layout that can produce a well formatted reference guide

Uses a subset of re-structured text (RST) markup:

\begin{enumerate}
  \item maintain readability in text editors
  \item allows for more advanced formatting to be inferred from simple markup
  \item allows the use of LaTeX for math
  \item bibtex citations
  \item use of sphinx directives to add warnings, notes etc.
\end{enumerate}

A reStructured text primer can be found \href{http://www.sphinx-doc.org/en/stable/rest.html}{here}

\end{frame}

\begin{frame}[fragile]{Numpy Style Docstrings: Functions}

\begin{minted}[fontsize=\tiny]{python}
  def function(param, keyword_param=True):
    """(#1) a short description of the function

    (#2) deprecation warnings

    (#3) extended summary

    (#4) Parameters
    ---------------
    param : type
            description of param 1
    keyword_param : boolean, optional(default=True)
                    description of keyword parameter
    
    (#5) Returns and Yields (explanation of returned values)
    Returns
    -------
    result

    (#6) Raises (optional section dealing with exceptions)

    (#7) See Also

    (#8) Notes and References

    (#9) Examples (can also be doctests)
    """
\end{minted}

\end{frame}

\begin{frame}[fragile]{Numpy Style Docstring: Typical Example}

\begin{minted}[fontsize=\small]{python}
def func(arg1, arg2=True):
    """Summary line.

    Extended description of function.

    Parameters
    ----------
    arg1 : int
        Description of arg1
    arg2 : bool, optional(default=True)
        Description of arg2

    Returns
    -------
    bool
        Description of return value

    """
\end{minted}

\end{frame}

\begin{frame}[fragile]{Numpy Style Docstrings: Classes}

\textbf{A) Class Docstring}

Use same sections as function (except Returns as \textbf{not} applicable)

The \mintinline{python}{__init__} constructor should be documented in the class docstring

An \textbf{Attributes} section can be located below Parameters to describe non-method attributes

A \textbf{Methods} section can be located below Attributes to document public methods

\textbf{B) Method Docstrings}

Documented in similar fashion to functions, but always exclude \mintinline{python}{self} from the list of parameters

\end{frame}

\begin{frame}{NumPy Style Docstring Guide and Example}

\textbf{NumPy Guide}

\url{https://github.com/numpy/numpy/blob/master/doc/HOWTO_DOCUMENT.rst.txt\#sections}

\textbf{Extended explanatory example (Napoleon sphinx extension)} with modules, functions, and classes  

\url{http://sphinxcontrib-napoleon.readthedocs.io/en/latest/example_numpy.html\#example-numpy}

\end{frame}


\begin{frame}[fragile]{Google Style Docstrings: Typical Example}

\textbf{Main difference} is use of indentation to separate sections rather than underlines

\begin{minted}[fontsize=\small]{python}
def func(arg1, arg2=True):
    """Summary line.

    Extended description of function.

    Args:
        arg1 (int): Description of arg1
        arg2 (bool, default=True): Description of arg2

    Returns:
        bool: Description of return value

    """
\end{minted}
\end{frame}


\begin{frame}{Google Style Docstrings}

\textbf{Extended explanatory example (Napoleon sphinx extension)} with module, functions and classes

\url{http://sphinxcontrib-napoleon.readthedocs.io/en/latest/example_google.html}

\end{frame}


\begin{frame}{Comparison: Numpy vs Google Style}

\begin{quote}
NumPy style tends to require more vertical space, whereas Google style tends to use more horizontal space. 
\end{quote}

\begin{quote}
Google style tends to be easier to read for short and simple docstrings, whereas NumPy style tends be easier to read for long and in-depth docstrings.
\end{quote}

\begin{flushright}
\url{http://www.sphinx-doc.org/en/stable/ext/napoleon.html\#google-vs-numpy}
\end{flushright}

\end{frame}

%END

\begin{frame}{My tips for writing good documentation}

\begin{enumerate}
  \item don't \textbf{duplicate} clearly written python code in your comments and documentation
  \item concise
  \item consistent
  \item include examples
  \item use descriptive variable names in your code to reduce comments
  \item make sure the comment has purpose
\end{enumerate}

\end{frame}

\begin{frame}{\href{http://www.sphinx-doc.org/en/stable/}{Sphinx} and Compilation Systems}

A benefit of using a standard for documentation is nice integration with build systems such as \href{http://www.sphinx-doc.org/en/stable/}{Sphinx}

You can document projects using \href{http://www.sphinx-doc.org/en/stable/ext/autodoc.html}{autodoc} which builds your documentation from internal docstrings

This is mainly useful for \textbf{larger} projects or code libraries

\end{frame}

\begin{frame}{Example: QuantEcon Project Documentation}

QuantEcon.py documentation can be found: \url{http://quanteconpy.readthedocs.io/en/latest/}

\end{frame}

\begin{frame}[fragile]{Creating a Sphinx Project}

\textbf{sphinx-apidoc}

To auto-generate sphinx rst files for a project \mintinline{bash}{sphinx-apidoc -o docs .} 

\textbf{sphinx-quickstart}

To start a sphinx documentation project it is best to use \mintinline{bash}{sphinx-quickstart}

\href{http://www.sphinx-doc.org/en/stable/tutorial.html}{Sphinx getting started tutorial}

\end{frame}

\begin{frame}{Resources}
\begin{enumerate}
\item \href{https://www.python.org/dev/peps/pep-0008/}{Python - PEP8}
\item \href{https://www.python.org/dev/peps/pep-0257/}{Python - Docstring Conventions}
\item \href{https://github.com/numpy/numpy/blob/master/doc/HOWTO_DOCUMENT.rst.txt}{NumpyDoc how-to Guide}
\item \href{http://sphinxcontrib-napoleon.readthedocs.io/en/latest/example_google.html}{Google Style Python Docstrings}
\item \href{http://www.sphinx-doc.org/en/stable/}{Sphinx}
\begin{itemize}
  \item \href{http://sphinxcontrib-napoleon.readthedocs.io/en/latest/sphinxcontrib.napoleon.html}{sphinxcontrib.napoleon package}
  \item \href{https://github.com/numpy/numpydoc}{Numpydoc}
  \item \href{http://www.sphinx-doc.org/en/stable/ext/autodoc.html}{autodoc}
\end{itemize}
\item \href{http://jupyter.org/}{Jupyter}
\end{enumerate}
\end{frame}

\section{Julia}

\begin{frame}{Julia Documentation}

Current state of the art: \href{https://github.com/JuliaDocs/Documenter.jl}{Documenter.jl}

Features:
\begin{enumerate}
  \item supports markdown
  \item support for LaTeX math
  \item Doctests, cross-references for docs, linter etc.
\end{enumerate}

documentation standards still to solidify

\end{frame}

\begin{frame}[fragile]{Julia Example}

  \begin{minted}[fontsize=\small]{julia}
    """
    add two numbers together

    # Keyword arguments:
    * `show:boolean`: show the resulting value (default=true)   
    """
    function add(a, b; show=true)
       result = a + b
       if show == true
         println("Result = ", result)
       end
       return result
     end

  \end{minted}

\end{frame}

\end{document}